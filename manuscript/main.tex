\documentclass[12pt]{article}
\newcommand\numberthis{\addtocounter{equation}{1}\tag{\theequation}}
\usepackage[margin=1in]{geometry} 
\usepackage{amsmath,amsthm,amssymb}
\usepackage{graphicx}
\usepackage{float}
\usepackage{listings}
\usepackage{hyperref}
\usepackage{dsfont}
\usepackage{xcolor, cancel}
\usepackage{enumerate}
\usepackage{algorithm2e}
\usepackage{bbm}
%\graphicspath{{../figs/}}
\newcommand{\N}{\mathbb{N}}
\newcommand{\Z}{\mathbb{Z}}
\newtheorem{theorem}{Theorem}[section]
\newtheorem{corollary}{Corollary}[theorem]
\newtheorem{lemma}[theorem]{Lemma}
%\newenvironment{theorem}[2][Theorem]{\begin{trivlist}
%\item[\hskip \labelsep {\bfseries #1}\hskip \labelsep {\bfseries #2.}]}{\end{trivlist}}
%\newenvironment{lemma}[2][Lemma]{\begin{trivlist}
%\item[\hskip \labelsep {\bfseries #1}\hskip \labelsep {\bfseries #2.}]}{\end{trivlist}}

\begin{document}
 
\title{Computing the Index of an Equilibrium Point and Persistent Homology}
\author{Aaron Havens}
\maketitle

\section{Preliminaries on Degree and Local Asymptotic Stability of a System}
\subsection{Topological Degree of a Differentiable Function}
Suppose $M$ and $N$ are two differentiable compact manifold of dimension $n$. For a regular point $x\in M$ the \textit{local degree} of a differentiable function $f: M \rightarrow N$ at $x$ is given by
\begin{align}
\text{deg}_x f = \text{sign} \left( \det \left( \frac{\partial f}{\partial x}\Big\rvert_{x} \right) \right)
\end{align}
By Sard's theorem, the image of all irregular points under $f$ has measure zero, moreover the set of regular values in $N$ is open and dense. We can consider the preimage of a regular value $y \in N$ by $f$ as
\begin{align*}
f^{-1}(y) = \{x\in M : f(x) = y\}.
\end{align*}
Because $M$ is compact, we know that the cardinality $|f^{-1}(y)|$ is finite. The \textit{degree} of $f$ the sum over the local degree of each $x \in f^{-1}(y)$ for a some regular value $y$ as
\begin{align*}\label{eq:top_deg}
\text{deg}(f) =\sum_{x\in f^{-1}(y)} \text{deg}_x(f). 
\end{align*}
Actually, the degree of $f$ is independent of choice of regular value $y$, the definition of degree is well-posed.
\subsection{Index of an Equilibrium Point}
Consider the differentiable function $f:\mathbb{R}^n \rightarrow \mathbb{R}^n$ and $\mathbb{R}^n$ equipped with the standard topology. Let $x^*\in\mathbb{R}^n$ be an isolated equilibrium point. Because $x^*$ is isolated, there exists $\varepsilon >0$ such that for the ball $B_{\varepsilon}(x^*)$ centered at $x^*$ there are no other equilibrium points. Let $S_{\varepsilon}(x^*)$ be the boundary of $B_{\varepsilon}(x^*)$. Then we can define the well-defined map 
\begin{align}
f_{\varepsilon} : S_{\varepsilon}(x^*) \rightarrow S^{n-1} : x \mapsto \frac{f(x)}{||f(x)||}.
\end{align}
We call the \textit{index} of the isolated equilibrium point $x^*$ to be the topological degree of $f_{\varepsilon}$ around $x^*$. Since both $S_{\varepsilon}(x^*)$ and $S^{n-1}$ are both compact differentiable manifolds and $f_{\varepsilon}$ is a differentiable on its domain, we can use the definition of degree of~\eqref{eq:top_deg} using some regular value $y \in S^{n-1}$.
\begin{align*}
	\text{ind}_{x^*}(f) = \text{deg} (f_{\varepsilon})
\end{align*}
The index of a map around an equilibrium point can be useful since it provides a necessary condition for local asymptotic stability.
\begin{theorem}[Lecture notes Theorem 6.7]
Consider the autonomous system
\begin{align*}
	\dot x = f(x)
\end{align*}
in dimension $n\neq 4$ with isolated equilibrium point $x^*$. Then $x^*$ is locally asymptotically stable only if the index of $f$ at $x^*$ is $(-1)^n$, that is
\begin{align*}
\textup{ind}_{x^*}(f) = (-1)^n.
\end{align*}
\end{theorem}
It should be noted that since degree holds under homotopy equivalence, its very possible that we have two vector fields homotopic to each other where one is stable and the other is not. Still the index can be of great use for nonlinear systems, but it is in general difficult to compute. In the following section we will propose some numerical methods for computing the index of a differentiable vector field in $\mathbb{R}^n$.
\section{Approximate Methods for Determining the Index of an Equilibrium Point}
Consider the problem of determining the index of an isolated equilibrium point $x^*$ of a function differentiable function $f:\mathbb{R}^d \rightarrow \mathbb{R}^d$. Defining $f_{\varepsilon} : S_{\varepsilon}(x^*) \rightarrow S^{n-1}$ as in the previous section, we are able to obtain samples $\{\left(x_n, f_{\varepsilon}(x_n)\right)\}_{n}$ for $x_n\in S_{\varepsilon}(x^*)$. In order to compute the index we must determine, for a regular value $y \in S^{n-1}$ of $f_{\varepsilon}$, the degree $f_{\varepsilon}$ and thus the preimage $f_{\varepsilon}^{-1}(y)$. Since we are only given samples of the function, we must find some way of numerically determining the preimage. Lets define the \textit{approximate preimage} of $f_{\varepsilon}$ at $y$ for $\delta>0$ to be
\begin{align}
N_{\delta}(y) := \{x\in S_{\varepsilon}(x^*) : ||f_{\varepsilon}(x) - y || < \delta \}
\end{align}
This allows to obtain an approximation of the preimage through sampling, as obtaining the true preimage would require possibly infinite samples. If we assume some regularity of the possible functions $f$ can take we can actually bound the number of samples necessary to ensure that each point in the preimage is contained in our sampled $\widehat{N}_{\delta}$. If we want to obtain $N_{\delta}(y)$ and $f_{\varepsilon}$ is $M$-Lipshitz, we would need to cover $S_{\varepsilon}(x^*)$ with enough points so that every point in $S_{\varepsilon}(x^*)$ is no more than $\delta/M$ away from a sampled point (i.e. an $\delta/M$-net of $S_{\varepsilon}(x^*)$ gives us an $\delta$-net of $S^{n-1}$ under the image of $f_{\varepsilon})$). However we may still run into the problem where we may always have other preimage points associated with nearby points in the image and it may require a $\delta$ very small, requiring more samples. We may have to query several preimages and compute some average degree to rule out these false preimage points that belong to the approximate preimage but not the exact preimage.

Still there remains the problem of what to do once $N_{\delta}(y)$ is obtained. How do we determine which points coorespond to a particular preimage point? Also, how do we choose a ``representative'' for each cluster of points belonging to preimage point?
\subsection{Clustering Approximate Preimages with Persistent Homology}
The task of determining the clusters corresponding to distinct points in the preimage may be thought of generally as a clustering problem. Its not obvious how to decide some local measure of association to assign a sample to a cluster. Rather than choosing an association distance heuristically, we can use a ideas of persistent homology (a slightly more advanced heuristic), specifically the $0d$-homology to sweep over an association distance and determine the number of connected components in our data set. Using the persistent diagram, we choose the appropriate distance based on ``births'' and ``deaths'' of connected components. If we choose $\delta$ small enough, we should be able to get isolate only neighborhoods corresponding the preimage $f_{\varepsilon}^{-1}(y)$.
\end{document}
